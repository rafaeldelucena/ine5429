%%% Template originaly created by Karol Kozioł (mail@karol-koziol.net) and modified for ShareLaTeX and
%%% Rafael de Lucena Valle use

\documentclass[a4paper,11pt]{article}

\usepackage[T1]{fontenc}
\usepackage[utf8]{inputenc}
\usepackage{graphicx}
\usepackage{xcolor}

\renewcommand\familydefault{\sfdefault}
\usepackage{tgheros}
\usepackage[defaultmono]{droidmono}

\usepackage{amsmath,amssymb,amsthm,textcomp}
\usepackage{enumerate}
\usepackage{multicol}
\usepackage{tikz}

\usepackage{geometry}
\geometry{total={210mm,297mm},
left=25mm,right=25mm,%
bindingoffset=0mm, top=20mm,bottom=20mm}


\linespread{1.3}

\newcommand{\linia}{\rule{\linewidth}{0.5pt}}

% custom theorems if needed
\newtheoremstyle{mytheor}
    {1ex}{1ex}{\normalfont}{0pt}{\scshape}{.}{1ex}
    {{\thmname{#1 }}{\thmnumber{#2}}{\thmnote{ (#3)}}}

\theoremstyle{mytheor}
\newtheorem{defi}{Definition}

% my own titles
\makeatletter
\renewcommand{\maketitle}{
\begin{center}
\vspace{2ex}
{\huge \textsc{\@title}}
\vspace{1ex}
\\
\linia\\
\@author \hfill \@date
\vspace{4ex}
\end{center}
}
\makeatother
%%%

% custom footers and headers
\usepackage{fancyhdr}
\pagestyle{fancy}
\lhead{}
\chead{}
\rhead{}
\cfoot{}
\rfoot{Page \thepage}
\renewcommand{\headrulewidth}{0pt}
\renewcommand{\footrulewidth}{0pt}
%

% code listing settings
\usepackage{listings}
\lstset{
    language=Python,
    basicstyle=\ttfamily\scriptsize,
    aboveskip={1.0\baselineskip},
    belowskip={1.0\baselineskip},
    columns=fixed,
    extendedchars=true,
    breaklines=true,
    tabsize=4,
    prebreak=\raisebox{0ex}[0ex][0ex]{\ensuremath{\hookleftarrow}},
    frame=lines,
    showtabs=false,
    showspaces=false,
    showstringspaces=false,
    keywordstyle=\color[rgb]{0.627,0.126,0.941},
    commentstyle=\color[rgb]{0.133,0.545,0.133},
    stringstyle=\color[rgb]{01,0,0},
    numbers=left,
    numberstyle=\small,
    stepnumber=1,
    numbersep=10pt,
    captionpos=t,
    escapeinside={\%*}{*)}
}

%%%----------%%%----------%%%----------%%%----------%%%

\begin{document}

\title{Teste de Primalidade}

\author{Rafael de Lucena Valle, Universidade Federal de Santa Catarina}

\date{10/10/2014}

\maketitle

\section*{Teste de Primalidade de Miller Rabin}
Números primos são muito utilizados em sistemas de segurança computacional, entretanto são somente úteis se tiverem centenas de dígitos. Não se conhece um método rápido para gerar diretamente números primos grandes, mas existem métodos para se verificar, alguns probabilísticos, se um dado número grande é primo. Dentre os probabilísticos, um dos mais conhecidos e usados é o teste de primalidade de Miller-Rabin, que utiliza as propriedades dos pseudoprimos de Fermat.


\section*{Algoritmo}
\begin{enumerate}
    \item Dado um numero inteiro impar n, represento-o de forma fatorada em $(2^r)*s + 1$ com $s$ sendo primo.
    \item Escolhe-se um numero randomico a numa faixa de 2 ate n-1.
    \item Testo as condições $a^s = 1 \mod n$ ou $a^2js = -1 \mod n$ para cada $j | 0 <= j <= r-1$
    \item Um número primo vai passar nos testes para todo valor a, se não encontrar nenhum valor nesta faixa é provavel que seja um número composto.
\end{enumerate}

\section*{Exemplos}

Para ilustrar a execução do algoritmo, vou mostrar um número composto 33 e o número primo 7.

\subsection*{Número composto}
Verificar se 33 é primo. decompondo 33 na forma de $(2^r)*s + 1$ com r ímpar:\\
Fazendo a decomposição de n para $(2^r)*s$:\\
32 é divido por 2, 5 vezes\\
$33 = (2^5)*1 + 1$, $s=1$ e $r=5$\\
a inteiro na faixa [2, n-1]\\
$a^s = 1 \mod n$, $a^1 = 1 \mod n$, $a^1 = 1 \mod 33$, $a^1  = 1$\\
$a^1 = 1$ não existe número inteiro maior que 1 que satisfaz\\
$a^(s*2*j) = n-1 \mod n$, $a^(2*2*j) = n-1 \mod n$, $a^4j = n - 1$\\
para $0 <= j < r$\\
$j=0$\\
$1 = n - 1$, $1 = 32$, $1 \neq 32$, $j=1$, $a^4 = 32$\\
$a = ??$ não existe número inteiro que satisfaz\\
$j=2$\\
$a^4 = 32$\\
$a = ??$ não existe número inteiro que satisfaz\\
$j=2$\\
$a^8 = 32$\\
$a = ??$ não existe número inteiro que satisfaz\\
$j=3$\\
$a^12 = 32$\\
$a = ??$ não existe número inteiro que satisfaz\\
$j=4$\\
$a^(2*2*4) = 32$, $a^16 = 32$\\
$a = ??$ não existe número inteiro que satisfaz, não foi encontrado nenhum valor que satisfaçam as condições, é o número 33 não é primo. Prova: $3 * 11 = 33$

\subsection*{Número Primo}
Testando um número primo $7 = 2^t*s + 1$\\
$6/s = 2^t$, $6/6 = 2^0$, $s = 6$ e $t = 0$\\
Fazendo os testes:
$a^s = 1 \mod n$, $a^6 = 1 \mod 7$, $1^6 = 1$\\
OK o número não é composto

\section*{Código}

\begin{lstlisting}[label={list:first},caption=Miller Rabin prime test em Python.]
import random
import sys

"""
Decompoe um numero par na forma (2^r) * s
"""
def decomposeBaseTwo(n):
    exponentOfTwo = 0
    while n % 2 == 0:
      n = n/2
      exponentOfTwo += 1
 
    return exponentOfTwo, n

"""
Verifica as condicoes
    Se (a^s === 1 (mod n) ou a^2js === -1 (mod n) 
    para um j | 0 <= j <= r-1
"""
def fillPrimeConditions(candidateNumber, p, exponent, remainder):
   candidateNumber = pow(candidateNumber, remainder, p)
 
   if candidateNumber == 1 or candidateNumber == p - 1:
      return False
 
   for _ in range(exponent):
      candidateNumber = pow(candidateNumber, 2, p)
 
      if candidateNumber == p - 1:
         return False
 
   return True
 
"""
  O numero randomico a na faixa que inicia em 2 pois, o teste 1^s = 1(mod n)
  Seria uma tentavia inutil
"""
def probablyPrime(p, accuracy=100):
   if p == 2 or p == 3: return True
   if p < 2: return False
 
   numTries = 0
   exponent, remainder = decomposeBaseTwo(p - 1)
 
   for _ in range(accuracy):
      candidateNumber = random.randint(2, p - 2)
      if fillPrimeConditions(candidateNumber, p, exponent, remainder):
         return False
 
   return True


def checkIsPrime():
    number = int(raw_input("Give some number to check if is prime: "))
    if (number == 1):
        print("\n\tOne is prime!\n")
        sys.exit()
    precision = raw_input("Which precision?: ")
    if (probablyPrime(number, int(precision))):
        print "\n\tThe number is probably prime!\n"
    else:
        print "\n\tThis is a compose number!\n"

def randomWithNDigits(n):
    range_start = 10**(n-1)
    range_end = (10**n)-1
    return random.randint(range_start, range_end)

def generateRandomPrime():
    digit = int(raw_input("Give the number of digits to create a random number: "))
    if digit < 1:
        print("\tMust be 1 digits or more!\n")
        return
    precision = int(raw_input("Which precision to test primality? "))
    random_number = randomWithNDigits(digit)
    while (probablyPrime(random_number, precision) == False):
        random_number = randomWithNDigits(digit);
    print "\tThe random number probably prime is: ", random_number, "\n\n"

def main():
    print("---- Miller Rabin Primality Test ----")
    functions = {
            'prime':checkIsPrime,
            'random':generateRandomPrime,
            'exit':sys.exit
            }
    
    while (1):
        print ("Type random will try generates a random or prime for check if is prime\n")
        type = raw_input("Choose random, prime or exit: ")
        if type not in functions.keys():
            print ("Invalid choice!")
            continue
        functions[type]()

if __name__ == '__main__':
    main()
\end{lstlisting}

\section*{Execução}

Para executar o script basta abrir um terminal e utilizar um shell tipo o bash, o script é interativo.

\begin{lstlisting}[label={list:second},caption=Executando o script.]
python miller_rabin.py

---- Miller Rabin Primality Test ----
Type random will try generates a random or prime for check if is prime

Choose random, prime or exit: random
Give the number of digits to create a random number: 200
Which precision to test primality? 100
    The random number probably prime is:  97427930850631007209932373111538240078238894443678593158471002592221782922131726207485466701550329045372339100590141645341390171807569749429398142665275255472653515797800907674306966481588454740955683
\end{lstlisting}

\end{document}
