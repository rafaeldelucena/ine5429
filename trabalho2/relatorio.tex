%%% Template originaly created by Karol Kozioł (mail@karol-koziol.net) and modified for ShareLaTeX and
%%% Rafael de Lucena Valle use

\documentclass[a4paper,11pt]{article}

\usepackage[T1]{fontenc}
\usepackage[utf8]{inputenc}
\usepackage{graphicx}
\usepackage{xcolor}

\renewcommand\familydefault{\sfdefault}
\usepackage{tgheros}
\usepackage[defaultmono]{droidmono}

\usepackage{amsmath,amssymb,amsthm,textcomp}
\usepackage{enumerate}
\usepackage{multicol}
\usepackage{tikz}

\usepackage{geometry}
\geometry{total={210mm,297mm},
left=25mm,right=25mm,%
bindingoffset=0mm, top=20mm,bottom=20mm}


\linespread{1.3}

\newcommand{\linia}{\rule{\linewidth}{0.5pt}}

% custom theorems if needed
\newtheoremstyle{mytheor}
    {1ex}{1ex}{\normalfont}{0pt}{\scshape}{.}{1ex}
    {{\thmname{#1 }}{\thmnumber{#2}}{\thmnote{ (#3)}}}

\theoremstyle{mytheor}
\newtheorem{defi}{Definition}

% my own titles
\makeatletter
\renewcommand{\maketitle}{
\begin{center}
\vspace{2ex}
{\huge \textsc{\@title}}
\vspace{1ex}
\\
\linia\\
\@author \hfill \@date
\vspace{4ex}
\end{center}
}
\makeatother
%%%

% custom footers and headers
\usepackage{fancyhdr}
\pagestyle{fancy}
\lhead{}
\chead{}
\rhead{}
\lfoot{Teste de Primalidade\textnumero{} 3}
\cfoot{}
\rfoot{Page \thepage}
\renewcommand{\headrulewidth}{0pt}
\renewcommand{\footrulewidth}{0pt}
%

% code listing settings
\usepackage{listings}
\lstset{
    language=Python,
    basicstyle=\ttfamily\small,
    aboveskip={1.0\baselineskip},
    belowskip={1.0\baselineskip},
    columns=fixed,
    extendedchars=true,
    breaklines=true,
    tabsize=4,
    prebreak=\raisebox{0ex}[0ex][0ex]{\ensuremath{\hookleftarrow}},
    frame=lines,
    showtabs=false,
    showspaces=false,
    showstringspaces=false,
    keywordstyle=\color[rgb]{0.627,0.126,0.941},
    commentstyle=\color[rgb]{0.133,0.545,0.133},
    stringstyle=\color[rgb]{01,0,0},
    numbers=left,
    numberstyle=\small,
    stepnumber=1,
    numbersep=10pt,
    captionpos=t,
    escapeinside={\%*}{*)}
}

%%%----------%%%----------%%%----------%%%----------%%%

\begin{document}

\title{Teste de Primalidade\textnumero{} 3}

\author{Rafael de Lucena Valle, Universidade Federal de Santa Catarina}

\date{26/05/2014}

\maketitle

\section*{Teste de Primalidade de Miller Rabin}
Números primos são muito utilizados em sistemas de segurança computacional. Entretanto, eles são somente úteis se tiverem centenas de dígitos. Não se conhece um método para gerar diretamente números primos grandes, mas existem métodos para se verificar, alguns probabilísticos, se um dado número grande é primo. Dentre os probabilísticos, um dos mais conhecidos e usados é o teste de primalidade de Miller-Rabin. Você deve desenvolver um programa em Java implementando este método. Deve ser possível, utilizando este programa, gerar números primos de pelos menos 100 dígitos decimais. Use a classe BigInteger para trabalhar com números grandes. Para gerar um número primo, primeiramente gere um número aleatório grande. Então, usando o Muller-Rabin, teste para ver se ele é provavelmente primo. Se não for, gere outro número aleatório, até que consiga um número primo.

\begin{lstlisting}[label={list:first},caption=Miller Rabin prime test em Python.]
import random
import sys

"""
Decompoe um numero par na forma (2^r) * s
"""
def decomposeBaseTwo(n):
    exponentOfTwo = 0
    while n % 2 == 0:
      n = n/2
      exponentOfTwo += 1
 
    return exponentOfTwo, n

"""
Verifica as condicoes
    Se (a^s === 1 (mod n) ou a^2js === -1 (mod n) 
    para um j | 0 <= j <= r-1
"""
def fillPrimeConditions(candidateNumber, p, exponent, remainder):
   candidateNumber = pow(candidateNumber, remainder, p)
 
   if candidateNumber == 1 or candidateNumber == p - 1:
      return False
 
   for _ in range(exponent):
      candidateNumber = pow(candidateNumber, 2, p)
 
      if candidateNumber == p - 1:
         return False
 
   return True
 
"""
  O numero randomico a na faixa que inicia em 2 pois, o teste 1^s = 1(mod n)
  Seria uma tentavia inutil
"""
def probablyPrime(p, accuracy=100):
   if p == 2 or p == 3: return True
   if p < 2: return False
 
   numTries = 0
   exponent, remainder = decomposeBaseTwo(p - 1)
 
   for _ in range(accuracy):
      candidateNumber = random.randint(2, p - 2)
      if fillPrimeConditions(candidateNumber, p, exponent, remainder):
         return False
 
   return True


def checkIsPrime():
    number = int(raw_input("Give some number to check if is prime: "))
    if (number == 1):
        print("\n\tOne is prime!\n")
        sys.exit()
    precision = raw_input("Which precision?: ")
    if (probablyPrime(number, int(precision))):
        print "\n\tThe number is probably prime!\n"
    else:
        print "\n\tThis is a compose number!\n"

def generateRandomPrime():
    bits = int(raw_input("Give the size of random number in bits: "))
    if bits < 2:
        print("\tMust be 2 bits or more!\n")
        return
    precision = int(raw_input("Which precision to test primality? "))
    random_number = random.getrandbits(bits)
    while (probablyPrime(random_number, precision) == False):
        random_number = random.getrandbits(bits)
    print "\tThe random number probably prime is: ", random_number, "\n\n"

def main():
    print("---- Miller Rabin Primality Test ----")
    functions = {
            'prime':checkIsPrime,
            'random':generateRandomPrime,
            'exit':sys.exit
            }
    
    while (1):
        print ("Type random will try generates a random or prime for check if is prime\n")
        type = raw_input("Choose random, prime or exit: ")
        if type not in functions.keys():
            print ("Invalid choice!")
            continue
        functions[type]()

if __name__ == '__main__':
    main()
\end{lstlisting}

Following Listing~\ref{list:first}\ldots{} 

\section*{Execução}

Para executar o script basta abrir um terminal e utilizar um shell tipo o bash, o script é interativo.

\begin{lstlisting}[label={list:second},caption=Executando o script.]
python miller_rabin.py
\end{lstlisting}

\end{document}
