%%% Template originaly created by Karol Kozioł (mail@karol-koziol.net) and modified for ShareLaTeX and
%%% Rafael de Lucena Valle use

\documentclass[a4paper,11pt]{article}

\usepackage[T1]{fontenc}
\usepackage[utf8]{inputenc}
\usepackage{graphicx}
\usepackage{xcolor}

\renewcommand\familydefault{\sfdefault}
\usepackage{tgheros}
\usepackage[defaultmono]{droidmono}

\usepackage{amsmath,amssymb,amsthm,textcomp}
\usepackage{enumerate}
\usepackage{multicol}
\usepackage{tikz}

\usepackage{geometry}
\geometry{total={210mm,297mm},
left=25mm,right=25mm,%
bindingoffset=0mm, top=20mm,bottom=20mm}


\linespread{1.3}

\newcommand{\linia}{\rule{\linewidth}{0.5pt}}

% custom theorems if needed
\newtheoremstyle{mytheor}
    {1ex}{1ex}{\normalfont}{0pt}{\scshape}{.}{1ex}
    {{\thmname{#1 }}{\thmnumber{#2}}{\thmnote{ (#3)}}}

\theoremstyle{mytheor}
\newtheorem{defi}{Definition}

% my own titles
\makeatletter
\renewcommand{\maketitle}{
\begin{center}
\vspace{2ex}
{\huge \textsc{\@title}}
\vspace{1ex}
\\
\linia\\
\@author \hfill \@date
\vspace{4ex}
\end{center}
}
\makeatother
%%%

% custom footers and headers
\usepackage{fancyhdr}
\pagestyle{fancy}
\lhead{}
\chead{}
\rhead{}
\cfoot{}
\rfoot{Page \thepage}
\renewcommand{\headrulewidth}{0pt}
\renewcommand{\footrulewidth}{0pt}
%

% code listing settings
\usepackage{listings}
\lstset{
    language=Python,
    basicstyle=\ttfamily\scriptsize,
    aboveskip={1.0\baselineskip},
    belowskip={1.0\baselineskip},
    columns=fixed,
    extendedchars=true,
    breaklines=true,
    tabsize=4,
    prebreak=\raisebox{0ex}[0ex][0ex]{\ensuremath{\hookleftarrow}},
    frame=lines,
    showtabs=false,
    showspaces=false,
    showstringspaces=false,
    keywordstyle=\color[rgb]{0.627,0.126,0.941},
    commentstyle=\color[rgb]{0.133,0.545,0.133},
    stringstyle=\color[rgb]{01,0,0},
    numbers=left,
    numberstyle=\scriptsize,
    stepnumber=1,
    numbersep=10pt,
    captionpos=t,
    escapeinside={\%*}{*)}
}

%%%----------%%%----------%%%----------%%%----------%%%

\begin{document}

\title{Diffie Helmann Key Exchange}

\author{Rafael de Lucena Valle, Universidade Federal de Santa Catarina}

\date{10/06/2014}

\maketitle

\section*{Troca de chaves Diffie Helmann}

A troca de chaves de Diffie-Hellman é um método de criptografia específico para troca de chaves desenvolvido por Whitfield Diffie e Martin Hellman e publicado em 1976. Utiliza a dificuldade de calcular de logaritmos discretos em um campo infinito.

O esquema de acordo de chaves Diffie-Helman não está limitado à interação entre apenas dois participantes. Qualquer número de usuários pode participar no acordo, executanto as interações do protocolo e trocando os dados intermediários entre si. 

\section*{Algoritmo}

\subsection*{Alice e Bob}
    Alice e Bob acordam em um número primo $p$ e uma raiz primitiva modulo $p$ maior que 2, ambos públicos. 
    
    Alice escolhe um número natural aleatório $a$ e envia $g^a$ para Bob.
    
    Bob também escolhe um número natural aleatório $b$ e envia $g^b$ para Alice.
    
    Alice calcula $(g^b)^a$.
    
    Bob calcula$(g^a)^b$.

    Alice quanto Bob possuem $g^{ab}$, que pode servir como a chave secreta compartilhada.
    
    Os valores de $(g^b)^a$ e $(g^a)^b$ são os mesmos porque os grupos são associativos a potência.
    
    Para decifrar uma mensagem $m$, enviada como $mg^{ab}$, deve primeiro calcular $(g^{ab})^{-1}$, da seguinte maneira:
    
    Bob sabe $|G|, b$, e $g^a$. Um resultado da teoria de grupos estabelece que a partir da construção de $G, x^{|G|} = 1$ para todo $x$ em $G$.
    
    Bob então calcula $(g^a)^{|G|-b} = g^{a(|G|-b)} = g^{a|G|-ab} = g^{a|G|}g^{-ab} = ((g^{|G|})^a)g^{-ab}=(1^a)g^{-ab}=g^{-ab}=(g^{ab})^{-1}$.
    
    Quando Alice envia a Bob a mensagem encriptada, $mg^{ab}$, Bob aplica $(g^{ab})^{-1}$ e recupera $mg^{ab}(g^{ab})^{-1} = m(1) = m$.

\subsection*{Alice Bob e Carol}

Alice, Bob e Carol participam de um acordo do tipo Diffie-Hellman de acordo com o exeomplo a seguir, onde todas as operações tomadas com módulo $p$:

    Os participantes selecionam os parâmetros iniciais do algoritmo $p$ e $g$;
    Os participantes geram suas respectivas chaves privadas, que chamaremos $a$, $b$ e $c$.
    Alice calcula $g^a$ e envia o resutlado para Bob.
    Bob calcula $(g^a)^b = g^{ab}$ e envia o resultado para Carol.
    Carol calcula $(g^{ab})^c = g^{abc}$ e utiliza esse valor como sua chave secreta compartilhada.
    Bob calcula $g^b$ e envia para Carol
    Carol calcula $(g^b)^c = g^{bc}$ e envia o resultado para Alice.
    Alice calcula $(g^{bc})^a = g^{bca} = g^{abc}$ e utiliza o resultado como chave secreta compartilhada.
    Carol calcula $g^c$ e envia para Alice.
    Alice calcula $(g^c)^a = g^{ca}$ e envia o resultado para Bob.
    Bob calcula $(g^{ca})^b = g^{cab} = g^{abc}$ e utiliza o resultado como chave secreta compartilhada.

Um intruso com acesso ao canal onde essas mensagens foram trocadas foi capaz de observar os valores $g^a, g^b, g^c, g^{ab}, g^{ac}$, e $g^{bc}$, mas não pode usar qualquer combinação desses valores para reproduzir $g^{abc}$.

\section*{Exemplos}

    Alice e Bob entram em acordo para usar um número primo $p=23$ e como base $g=5$.\\
    Alice escolhe um inteiro secreto $a=6$, e então envia a Bob $A = g^a \mod p$.\\
        $A = 5^6 \mod 23$\\
        $A = 8$\\
    Bob escolhe um inteiro secreto $b=15$, e então envia a Alice $B = g^b \mod p$\\
        $B = 5^{15} \mod 23$\\
        $B = 19$\\
    Alice calcula $s = B^a \mod p$\\
        $s = 19^6 \mod 23$\\
        $s = 2$\\
    Bob calcula $s = A^b \mod p$\\
        $s = 8^{15} \mod 23$\\
        $s = 2$\\

    Alice e Bob compartilham agora uma chave secreta: s = 2. Isto é possível porque 6*15 é o mesmo que 15*6. Alguém que tenha descoberto estes dois inteiros privados também será capaz de calcular s da seguinte maneira:\\
        $s = 5^{6*15} \mod 23$\\
        $s = 5^{15*6} \mod 23$\\
        $s = 5^{90} \mod 23$\\
        $s = 2$\\

\clearpage
\section*{Código}

\begin{lstlisting}[caption=Diffie Helmann utilizando Miller Rabin em Python.]
import random
import sys
import math

"""
Decompoe um numero par na forma (2^r) * s
"""
def decomposeBaseTwo(n):
    exponentOfTwo = 0
    while n % 2 == 0:
      n = n/2
      exponentOfTwo += 1
 
    return exponentOfTwo, n

"""
Verifica as condicoes
    Se (a^s === 1 (mod n) ou a^2js === -1 (mod n) 
    para um j | 0 <= j <= r-1
"""
def fillPrimeConditions(candidateNumber, p, exponent, remainder):
   candidateNumber = pow(candidateNumber, remainder, p)
 
   if candidateNumber == 1 or candidateNumber == p - 1:
      return False
 
   for _ in range(exponent):
      candidateNumber = pow(candidateNumber, 2, p)
 
      if candidateNumber == p - 1:
         return False
 
   return True
 
"""
  O numero randomico a na faixa que inicia em 2 pois, o teste 1^s = 1(mod n)
  Seria uma tentavia inutil
"""
def probablyPrime(p, accuracy=100):
   if p == 2 or p == 3: return True
   if p < 2: return False
 
   numTries = 0
   exponent, remainder = decomposeBaseTwo(p - 1)
 
   for _ in range(accuracy):
      candidateNumber = random.randint(2, p - 2)
      if fillPrimeConditions(candidateNumber, p, exponent, remainder):
         return False
 
   return True

def generateRandomPrime(bits, precision):
    random_number = random.getrandbits(bits)
    while (probablyPrime(random_number, precision) == False):
        random_number = random.getrandbits(bits)
    return random_number


def bigRange(start, stop, step):
    i = start
    while i < stop:
        yield i
        i += step

def primeFactors(n):
    f, fs = 3, set()
    while n % 2 == 0:
        fs.add(2)
        n /= 2
    while f * f <= n:
        while n % f == 0:
            fs.add(f)
            n /= f
        f += 2
    if n > 1: fs.add(n)
    return fs

def primitiveRoots(number):
    if number:
        if number == 2:
            yield 1
        if number == 3:
            yield 2
        phi = number - 1
        factors = primeFactors(phi)
        for m in bigRange(2, phi, 1):
            is_root = True
            for p in factors:
                if pow(m, int(phi / p), number) == 1:
                    is_root = False
            if (is_root):
                yield m

def diffie_helmann(bits, precision):
    prime = generateRandomPrime(bits, precision)
    for root in primitiveRoots(prime):
        if root == 2:
            break
        XA = random.getrandbits(bits - 1)
        XB = random.getrandbits(bits - 1)
        YA = pow(root, XA, prime)
        YB = pow(root, XB, prime)

        K1 = pow(YB, XA, prime)
        K2 = pow(YA, XB, prime)
        return


def main():
    bits = int(sys.argv[1])
    precision = int (sys.argv[2])
    diffie_helmann(bits, precision)

if __name__ == '__main__':
    main()

\end{lstlisting}

\section*{Execução}

Para executar o script basta abrir um terminal e utilizar um shell tipo o bash, o script é interativo.

\begin{lstlisting}[label={list:second},caption=Executando o script.]
python diffie_helmann.py
\end{lstlisting}

\end{document}
