%%% Template originaly created by Karol Kozioł (mail@karol-koziol.net) and modified for ShareLaTeX and
%%% Rafael de Lucena Valle use

\documentclass[a4paper,11pt]{article}

\usepackage[T1]{fontenc}
\usepackage[utf8]{inputenc}
\usepackage{graphicx}
\usepackage{xcolor}

\renewcommand\familydefault{\sfdefault}
\usepackage{tgheros}
\usepackage[defaultmono]{droidmono}

\usepackage{amsmath,amssymb,amsthm,textcomp}
\usepackage{enumerate}
\usepackage{multicol}
\usepackage{tikz}

\usepackage{geometry}
\geometry{total={210mm,297mm},
left=25mm,right=25mm,%
bindingoffset=0mm, top=20mm,bottom=20mm}


\linespread{1.3}

\newcommand{\linia}{\rule{\linewidth}{0.5pt}}

% custom theorems if needed
\newtheoremstyle{mytheor}
    {1ex}{1ex}{\normalfont}{0pt}{\scshape}{.}{1ex}
    {{\thmname{#1 }}{\thmnumber{#2}}{\thmnote{ (#3)}}}

\theoremstyle{mytheor}
\newtheorem{defi}{Definition}

% my own titles
\makeatletter
\renewcommand{\maketitle}{
\begin{center}
\vspace{2ex}
{\huge \textsc{\@title}}
\vspace{1ex}
\\
\linia\\
\@author \hfill \@date
\vspace{4ex}
\end{center}
}
\makeatother
%%%

% custom footers and headers
\usepackage{fancyhdr}
\pagestyle{fancy}
\lhead{}
\chead{}
\rhead{}
\cfoot{}
\rfoot{Page \thepage}
\renewcommand{\headrulewidth}{0pt}
\renewcommand{\footrulewidth}{0pt}
%

% code listing settings
\usepackage{listings}
\lstset{
    language=Python,
    basicstyle=\ttfamily\scriptsize,
    aboveskip={1.0\baselineskip},
    belowskip={1.0\baselineskip},
    columns=fixed,
    extendedchars=true,
    breaklines=true,
    tabsize=4,
    prebreak=\raisebox{0ex}[0ex][0ex]{\ensuremath{\hookleftarrow}},
    frame=lines,
    showtabs=false,
    showspaces=false,
    showstringspaces=false,
    keywordstyle=\color[rgb]{0.627,0.126,0.941},
    commentstyle=\color[rgb]{0.133,0.545,0.133},
    stringstyle=\color[rgb]{01,0,0},
    numbers=left,
    numberstyle=\scriptsize,
    stepnumber=1,
    numbersep=10pt,
    captionpos=t,
    escapeinside={\%*}{*)}
}

%%%----------%%%----------%%%----------%%%----------%%%

\begin{document}

\title{Cryptography with RSA}

\author{Rafael de Lucena Valle, Universidade Federal de Santa Catarina}

\date{16/06/2014}

\maketitle

\section*{Criptografia por RSA}

\section*{Algoritmo}

Geração de Chaves

1) Escolho dois números primos grandes, $P$ and $Q$.
Fazemos isso utilizando utilizando a geração de números randômicos no Python e o teste de primalidade de Miller Rabin.

2) Calculamos $N = P*Q$

3) Calculamos $M$ sendo o totiente de $P$ * totiente $Q$, assim $M = (P - 1)*(Q - 1)$

4) Escolhemos um número $E$ coprimo a $M$, ou seja o máximo divisor comum igual a 1, utilizamos o algoritmo de Euclides. Para fazer estes calculos com números grandes é necessário utilizar o algoritmo extendido de Euclides, que recebe dois inteiros positivos $a, b$ as e retorna uma tripla $(g, x, y)$, tal que $ax + by = g = gcd(a, b)$.

5) Encontro $D$, tal que $D \% M = 1$, isto é equivalente a encontrar $D$ que satisfaça $D*E = 1 + X*M$, aonde $X$ é qualquer inteiro. Podemos reescrever a fórmula de modo $D = (1 + X*M) / E$.

Ao final $D$ será a chave privada e $E$ será a chave pública. O número a ser criptografado deve ser menor que $P$ e $Q$.\\

Para Criptografar H:\\
$C = H^E \% N$\\

Para Decriptar:\\
$H = C^D \% N$

\section*{Exemplos}

$P = 3$\\
$Q = 11$\\
$N = 3 * 11 = 33$\\
$M = (3 - 1)(11 - 1) = 20$\\

Encontramos E, tal que $MDC(E, M) = 1$\\
para $E = 2$:\\
    => $MDC(2, 20) = 2$\\
para $E = 3$:\\
    => $MDC(3, 20) = 1$ (3 é coprimo a 20)\\

Encontramos $D = (1 + X*M) / E$.\\
para $X = 0$:\\
    => $D = 1 / 3$\\
para $X = 1$:\\
    => $D = 21 / 3$\\
    => $D = 7$\\

Criptografando a mensagem $Z = 2$:\\
$C = Z^E \% N$\\
$C = 2^3 \% 33 = 8$\\

Decriptando:\\
$Z = C^D \% N$\\
$2 = 8^7 \% 33 = 2$

\section*{Código}

\begin{lstlisting}[caption=RSA utilizando Miller Rabin em Python.]
import random
import math
import copy

"""
Decompoe um numero par na forma (2^r) * s
"""
def decomposeBaseTwo(n):
    exponentOfTwo = 0
    while n % 2 == 0:
      n = n/2
      exponentOfTwo += 1
 
    return exponentOfTwo, n

"""
Verifica as condicoes
    Se (a^s === 1 (mod n) ou a^2js === -1 (mod n) 
    para um j | 0 <= j <= r-1
"""
def fillPrimeConditions(candidateNumber, p, exponent, remainder):
   candidateNumber = pow(candidateNumber, remainder, p)
 
   if candidateNumber == 1 or candidateNumber == p - 1:
      return False
 
   for _ in range(exponent):
      candidateNumber = pow(candidateNumber, 2, p)
 
      if candidateNumber == p - 1:
         return False
 
   return True
 
"""
  O numero randomico a na faixa que inicia em 2 pois, o teste 1^s = 1(mod n)
  Seria uma tentavia inutil
"""
def probablyPrime(p, accuracy=100):
   if p == 2 or p == 3: return True
   if p < 2: return False
 
   numTries = 0
   exponent, remainder = decomposeBaseTwo(p - 1)
 
   for _ in range(accuracy):
      candidateNumber = random.randint(2, p - 2)
      if fillPrimeConditions(candidateNumber, p, exponent, remainder):
         return False
 
   return True


'''returns the Greatest Common Divisor of a and b'''
def euclid(a,b):
    a = abs(a)
    b = abs(b)
    if a < b:
        a, b = b, a

    while b != 0:
        a, b = b, a % b

    return a

'''returns 'True' if the values in the list L are all co-prime
   otherwier, it returns 'False'. '''

def coprime(L):
    for i in range (0, len(L)):
        for j in range (i + 1, len(L)):
            if euclid(L[i], L[j]) != 1:
                return False

    return True

def extendedEuclid(a, b):
    x,y, u,v = 0,1, 1,0
    while a != 0:
        q,r = b//a,b%a; m,n = x-u*q,y-v*q
        b,a, x,y, u,v = a,r, u,v, m,n
    return b, x, y

'''returns the multiplicative inverse of a in modulo m as a positve value between zero and m-1'''
def multiplicativeInverse(a, m):
    if coprime([a, m]) == False:
        return 0
    else:
        linearcombination = extendedEuclid(a, m)
        return linearcombination[1] % m


def generateRandomPrime(bits, precision):
    random_number = random.getrandbits(bits)
    while (probablyPrime(random_number, precision) == False):
        random_number = random.getrandbits(bits)
    return random_number

''' Try to find a large pseudo primes and generate public and private keys for RSA encryption.'''
def generateKeys(a,k):
    p = generateRandomPrime(a, k)
    while True:
        q = generateRandomPrime(a, k)
        if q != p:
            break
    
    n = p * q
    m = (p-1) * (q-1)
    while True:
        e = random.randint(1, m)
        if coprime([e, m]):
            break
    d = multiplicativeInverse(e, m)
    return (n, e, d)

'''Converts a string to a list of integers based on ASCII values, printable characters range is 0x20 - 0x7E.'''
def string2numList(strn):
    returnList = []
    for chars in strn:
        returnList.append(ord(chars))
    return returnList

'''Converts a list of integers to a string based on ASCII values'''
def numList2string(L):
    returnList = []
    returnString = ''
    for nums in L:
        returnString += chr(nums)
    return returnString

'''Take a list of integers(each between 0 and 127), and combines them into block size n using base 256. If len(L) % n != 0, use some random junk to fill L to make it '''
def numList2blocks(L,n):
    returnList = []
    toProcess = copy.copy(L)
    if len(toProcess) % n != 0:
        for i in range (0, n - len(toProcess) % n):
            toProcess.append(random.randint(32, 126))
    for i in range(0, len(toProcess), n):
        block = 0
        for j in range(0, n):
            block += toProcess[i + j] << (8 * (n - j - 1))
        returnList.append(block)
    return returnList

'''inverse function of numList2blocks.'''
def blocks2numList(blocks,n):
    toProcess = copy.copy(blocks)
    returnList = []
    for numBlock in toProcess:
        inner = []
        for i in range(0, n):
            inner.append(numBlock % 256)
            numBlock >>= 8
        inner.reverse()
        returnList.extend(inner)
    return returnList

'''given a string message, public keys and blockSize, encrypt using RSA algorithms.'''
def encrypt(message, modN, e, blockSize):
    cipher = []
    numList = string2numList(message)
    numBlocks = numList2blocks(numList, blockSize)
    for blocks in numBlocks:
        cipher.append(pow(blocks, e, modN))
    return cipher

'''reverse function of encrypt'''
def decrypt(secret, modN, d, blockSize):
    numBlocks = []
    numList = []
    for blocks in secret:
        numBlocks.append(pow(blocks, d, modN))
    numList = blocks2numList(numBlocks, blockSize)
    message = numList2string(numList)
    return message

if __name__=='__main__':
    bits = int(raw_input("Give the size of random number in bits: "))
    precision = int(raw_input("Which precision to test primality? "))
    message = raw_input("Which message to be used? ")
    (n, e, d) = generateKeys(bits, precision)
    print ('n = {0}'.format(n))
    print ('Public Key e = {0}'.format(e))
    print ('Private Key d = {0}'.format(d))
    cipher = encrypt(message, n, e, len(message))
    print(cipher)
    Amessage = decrypt(cipher, n, d, len(message))
    print('The decrypt message is: ', Amessage)
\end{lstlisting}

\section*{Execução}

Para executar o script basta abrir um terminal e utilizar um shell tipo o bash, o script é interativo.

\begin{lstlisting}[label={list:second},caption=Executando o script.]
python rsa.py
\end{lstlisting}

\end{document}
