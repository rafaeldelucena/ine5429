%%% Template originaly created by Karol Kozioł (mail@karol-koziol.net) and modified for ShareLaTeX and
%%% Rafael de Lucena Valle use

\documentclass[a4paper,11pt]{article}

\usepackage[T1]{fontenc}
\usepackage[utf8]{inputenc}
\usepackage{graphicx}
\usepackage{xcolor}

\renewcommand\familydefault{\sfdefault}
\usepackage{tgheros}
\usepackage[defaultmono]{droidmono}

\usepackage{amsmath,amssymb,amsthm,textcomp}
\usepackage{enumerate}
\usepackage{multicol}
\usepackage{tikz}

\usepackage{geometry}
\geometry{total={210mm,297mm},
left=25mm,right=25mm,%
bindingoffset=0mm, top=20mm,bottom=20mm}


\linespread{1.3}

\newcommand{\linia}{\rule{\linewidth}{0.5pt}}

% custom theorems if needed
\newtheoremstyle{mytheor}
    {1ex}{1ex}{\normalfont}{0pt}{\scshape}{.}{1ex}
    {{\thmname{#1 }}{\thmnumber{#2}}{\thmnote{ (#3)}}}

\theoremstyle{mytheor}
\newtheorem{defi}{Definition}

% my own titles
\makeatletter
\renewcommand{\maketitle}{
\begin{center}
\vspace{2ex}
{\huge \textsc{\@title}}
\vspace{1ex}
\\
\linia\\
\@author \hfill \@date
\vspace{4ex}
\end{center}
}
\makeatother
%%%

% custom footers and headers
\usepackage{fancyhdr}
\pagestyle{fancy}
\lhead{}
\chead{}
\rhead{}
\cfoot{}
\rfoot{Page \thepage}
\renewcommand{\headrulewidth}{0pt}
\renewcommand{\footrulewidth}{0pt}
%

% code listing settings
\usepackage{listings}
\lstset{
    language=Python,
    basicstyle=\ttfamily\scriptsize,
    aboveskip={1.0\baselineskip},
    belowskip={1.0\baselineskip},
    columns=fixed,
    extendedchars=true,
    breaklines=true,
    tabsize=4,
    prebreak=\raisebox{0ex}[0ex][0ex]{\ensuremath{\hookleftarrow}},
    frame=lines,
    showtabs=false,
    showspaces=false,
    showstringspaces=false,
    keywordstyle=\color[rgb]{0.627,0.126,0.941},
    commentstyle=\color[rgb]{0.133,0.545,0.133},
    stringstyle=\color[rgb]{01,0,0},
    numbers=left,
    numberstyle=\small,
    stepnumber=1,
    numbersep=10pt,
    captionpos=t,
    escapeinside={\%*}{*)}
}

%%%----------%%%----------%%%----------%%%----------%%%

\begin{document}

\title{Raiz primitiva módulo P}

\author{Rafael de Lucena Valle, Universidade Federal de Santa Catarina}

\date{06/10/2014}

\maketitle

\section*{Conceito}
$G$ é Raiz primitiva módulo $P$, se para cada número $Y$ coprimo a $P$ (que possuem máximo divisor comum = 1) seja congruente (se não satisfazer as condições iniciais, a diferença de $A$ menos $B$ é um inteiro múltipo de $P$ (ou $P$ divide $A$ menos $B$)) a uma potência de $G$ módulo $P$.

A diferença de uma raiz primitiva no módulo $P$ para um número qualquer, é que para valores $[1, P] \bmod P$ de um número qualquer os números podem se repetir, o que não é uma característica segura para usá-la em handshakes de comunicação.
\section*{Algoritmo}
O algoritmo utilizado foi:

\begin{enumerate}
\item Dado um número de entrada primo $P$, teremos $N=P-1$ números coprimos a $P$, nas quais o $MDC(Ni,P)$ igual a 1.
\item É necessário determinar todos os fatores primos $Fi$ de $N$.
\item Então, escolha um número $A$ e calcule $X = A^{N/Fi} \bmod P$ para todos os fatores primos encontrados.
\item Se encontrar $X$ igual a $1$, então $A$ não é uma raiz primitiva, do contrário é.
\end{enumerate}

\section*{Exemplos}
\subsection*{Exemplo}
3 é raiz primitiva módulo 7 pois:\\
$3^1 \bmod7 = 3$\\
$3^2 \bmod 7 = 2$\\
$3^3 \bmod 7 = 6$\\
$3^4 \bmod 7 = 4$\\
$3^5 \bmod 7 = 5$\\
$3^6 \bmod 7 = 6$\\
\subsection*{Contraexemplo}
$4^1 \bmod 5 = 4$\\
$4^2 \bmod 5 = 1$\\
$4^3 \bmod 5 = 4$\\
$4^4 \bmod 5 = 1$\\

Foram encontrados X igual a 1, que não satisfazem as condições iniciais, logo 4 não é uma raiz primitiva módulo P de 5.
\section*{Código}

\begin{lstlisting}[label={list:first},caption=Encontrando Raiz Primitiva Módulo P em Python.]
import math
import sys

def bigRange(start, stop, step):
    i = start
    while i < stop:
        yield i
        i += step

def primeFactors(n):
    f, fs = 3, set()
    while n % 2 == 0:
        fs.add(2)
        n /= 2
    while f * f <= n:
        while n % f == 0:
            fs.add(f)
            n /= f
        f += 2
    if n > 1: fs.add(n)
    return fs

def primitiveRoots(number):
    roots = []
    if (number < 2):
        return roots
    phi = number - 1
    factors = primeFactors(phi)
    for m in bigRange(2, phi, 1):
        is_root = True
        for p in factors:
            if pow(m, int(phi / p), number) == 1:
                is_root = False
        if (is_root):
            roots.append(m)
    return roots

def main():
    number = int(raw_input("Enter a prime number: "))
    roots = primitiveRoots(number)
    if (roots):
        print("Find ", len(roots), " primitive roots module ", number, " = ", roots)

if __name__ == "__main__":
    sys.exit(main())
\end{lstlisting}

\section*{Execução}
Implementei um Script em Python para gerar as raizes primitives de um numero primo P. Para executar basta extrair o arquivo e executar os comandos exibidos abaixo.

\begin{lstlisting}[label={list:second},caption=Executando o script.]
python primitive_root.py
\end{lstlisting}

\end{document}
